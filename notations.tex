Before diving into the details, let's establish the notations used throughout this tutorial. The clarity in notation is crucial for better understanding. Here are the key notations:

\begin{itemize}
    \item $X_t$: The asset price at time $t$.
    \item $T$: The maturity or exercise time of the option.
    \item $g(x)$: The option's pay-off function is a function of the underlying asset's value $x$.
    \item $\mu(t, X_t)$: The drift term in the asset price SDE.
    \item $\sigma(t, X_t)$: The volatility term in the asset price SDE.
    \item $dW_t$: The Wiener process or Brownian motion.
    \item $u(t, x)$: The option price at time $t$, given the current value of the underlying asset $X_t = x$.
    \item For $\alpha = (\alpha_1, \ldots, \alpha_d) \in \mathbb{N}_0^d$ multiindex, and $G \subset \mathbb{R}^d$ (open): set $|\alpha| = \sum_{i=1}^{d} \alpha_i$.
    \item For smooth $u : G \rightarrow \mathbb{R}$ and $x = (x_1, \ldots, x_d) \in G$, define:
    \[
    D^\alpha u(x) := \frac{{\partial^{|\alpha|} u(x)}}{{\partial x_1^{\alpha_1} \cdots \partial x_d^{\alpha_d}}} = {\partial^{\alpha_1}_{x_1}} \cdots {\partial^{\alpha_d}_{x_d}} u(x).
    \]
    \item Elements of $G \subset \mathbb{R}^d$ are points $x = (x_1, \ldots, x_d)$.
    \item For $k \in \mathbb{N}_0$, $D^k u(x) := \{D^\alpha u(x) : |\alpha| = k\}$ is the set of all partial derivatives of order $k$.
\end{itemize}

These notations will be used consistently throughout the tutorial to ensure clarity and consistency in the explanations and examples.

Now that we have established the notations let's delve deeper into the specific numerical methods for solving PDEs in the context of option pricing.
