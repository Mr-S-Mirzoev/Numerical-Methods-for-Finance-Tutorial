\subsubsection{Heat Equation}

The heat equation describes the heat distribution in a given region over time. It can be represented as a system of equations as follows:

\begin{equation*}
\begin{cases}
\partial_t u - \partial_{xx} u = f(t, x) & \text{in } J \times G, \\
u = 0 & \text{on } J \times \partial G, \\
u(0, \cdot) = u_0 & \text{in } G,
\end{cases}
\end{equation*}

Where:
\begin{itemize}
\item The equation $u(0, \cdot) = u_0$ in $G$ represents the \textcolor{blue}{initial condition}.
\item The equation $u = 0$ on $J \times \partial G$ represents the \textcolor{blue}{boundary condition}. Here, it is of Dirichlet type and homogeneous.
\end{itemize}

Such partial differential equations are known as initial-boundary value problems.

\subsubsection{Discretization of the PDE}

To solve the heat equation numerically, we discretize the computational domain $J \times G$ using a \textbf{discrete grid}. The grid is defined as follows:

\begin{equation*}
\{(t_m, x_i)\}, \quad i = 0, \ldots, N+1, \quad m = 0, \ldots, M,
\end{equation*}

where $x_i$ are the \textcolor{blue}{spatial grid points} with a \textcolor{blue}{spacing (or space step size)} of $h$, and $t_m$ are the \textcolor{blue}{time levels} with a \textcolor{blue}{time step size} of $k$.

\medskip

The spatial grid points $x_i$ are determined by the interval $G = (a, b)$ and the number of grid points $N$ as $x_i = a + ih$, where $h = \frac{b-a}{N+1}$. 

The time levels $t_m$ are determined by the interval $J = (0, T)$ and the number of time steps $M$ as $t_m = mk$, where $k = \frac{T}{M}$.

\medskip

Next, we represent the exact solution $u(t, x)$ by its values on the grid:

\begin{equation*}
u(t, x) \rightarrow \{u_{m,i} = u(t_m, x_i)\}, \quad i = 0, \ldots, N+1, \quad m = 0, \ldots, M.
\end{equation*}

\textcolor{blue}{The goal is to approximate the values $\{u^{m}_{i}\}$}. Values of the solution between grid points are then found \textbf{using some interpolation method}.