\subsubsection{Heat Equation}

The heat equation describes the heat distribution in a given region over time. It can be represented as a system of equations as follows:

\begin{equation*}
\begin{cases}
\partial_t u - \partial_{xx} u = f(t, x) & \text{in } J \times G, \\
u = 0 & \text{on } J \times \partial G, \\
u(0, \cdot) = u_0 & \text{in } G,
\end{cases}
\end{equation*}

Where:
\begin{itemize}
\item The equation $u(0, \cdot) = u_0$ in $G$ represents the \textcolor{blue}{initial condition}.
\item The equation $u = 0$ on $J \times \partial G$ represents the \textcolor{blue}{boundary condition}. Here, it is of Dirichlet type and homogeneous.
\end{itemize}

Such partial differential equations are known as initial-boundary value problems.

\subsubsection{Discretization of the PDE}

To solve the heat equation numerically, we discretize the computational domain $J \times G$ using a \textbf{discrete grid}. The grid is defined as follows:

\begin{equation*}
\{(t_m, x_i)\}, \quad i = 0, \ldots, N+1, \quad m = 0, \ldots, M,
\end{equation*}

where $x_i$ are the \textcolor{blue}{spatial grid points} with a \textcolor{blue}{spacing (or space step size)} of $h$, and $t_m$ are the \textcolor{blue}{time levels} with a \textcolor{blue}{time step size} of $k$.

\medskip

The spatial grid points $x_i$ are determined by the interval $G = (a, b)$ and the number of grid points $N$ as $x_i = a + ih$, where $h = \frac{b-a}{N+1}$. 

The time levels $t_m$ are determined by the interval $J = (0, T)$ and the number of time steps $M$ as $t_m = mk$, where $k = \frac{T}{M}$.

\medskip

Next, we represent the exact solution $u(t, x)$ by its values on the grid:

\begin{equation*}
u(t, x) \rightarrow \{u_{m,i} = u(t_m, x_i)\}, \quad i = 0, \ldots, N+1, \quad m = 0, \ldots, M.
\end{equation*}

\textcolor{blue}{The goal is to approximate the values $\{u^{m}_{i}\}$}. Values of the solution between grid points are then found \textbf{using some interpolation method}.

\subsubsection{The Finite Difference Method (FDM)}

The Finite Difference Method (FDM) is a commonly used numerical method for solving partial differential equations. It approximates the derivatives of a function using only its values on the grid. Let's start by defining the difference quotients, also known as finite differences.

\textbf{Difference Quotients (= Finite Differences)}

Consider a function $g(x)$ of one variable. Assume that $g \in C^2$. Using Taylor's formula, we have:

\begin{equation*}
g' (x) = \frac{g(x + h) - g(x)}{h} - \frac{h}{2} g''(\xi), \quad \xi \in [x, x + h].
\end{equation*}

If $g_i = g(x_i)$ are the values of $g$ on the grid $\{x_i\}$, we obtain:

\begin{equation*}
g' (x_i) = \frac{g_{i+1} - g_i}{h} + O(h) =: {(\delta^{+}_{x} g)}_i + O(h).
\end{equation*}

Similarly, for $g \in C^4$:

\begin{equation*}
g'' (x_i) = \frac{g_{i+1} - 2g_i + g_{i-1}}{h^2} + O(h^2) =: {(\delta_{xx}g)}_i + O(h^2).
\end{equation*}

\subsubsubsection{FD Scheme}

In the Finite Difference Method (FDM), we use a finite difference scheme to replace the partial differential equation $\partial_t u - \partial_{xx}u = f$ with a set of algebraic equations.

\textbf{FD Scheme}

Let $\theta \in [0, 1]$. Then, we define the following set of algebraic equations:

\[
\begin{cases}
\mathcal{E}^{m}_{i} = \theta f^{m+1}_{i} + (1 - \theta) f^{m}_{i} & \text{for } i = 1, \ldots, N, \ m = 0, \ldots, M-1, \\
u^{0}_{i} = u_{0,i} & \text{for } i = 1, \ldots, N, \\
u^{m}_{k} = 0 & \text{for } k \in \{0, N+1\}, \ m = 0, \ldots, M,
\end{cases}
\]

where $\mathcal{E}^{m}_{i}$ represents the finite difference operator:

\begin{align*}
\mathcal{E}^{m}_{i} := & k^{-1} (u^{m+1}_{i} - u^{m}_{i}) - [\theta(\delta_{xx}u)^{m+1}_{i} + (1 - \theta)(\delta_{xx}u)^{m}_{i}] \\
= & \frac{u^{m+1}_{i} - u^{m}_{i}}{k} - \\
- & \left[\theta \frac{(u^{m+1}_{i+1} - 2u^{m+1}_{i} + u^{m+1}_{i-1})}{h^2} + (1 - \theta) \frac{(u^{m}_{i+1} - 2u^{m}_{i} + u^{m}_{i-1})}{h^2} \right].
\end{align*}

This finite difference scheme represents the discretization of the partial differential equation. The equation $\mathcal{E}^{m}_{i} = 0$ defines the values of the solution $u^{m}_{i}$ at different time levels and spatial grid points, as well as the given source term $f$. By solving this set of algebraic equations iteratively, we can approximate the solution $u(t, x)$ of the original partial differential equation.

\subsubsubsection{FD Scheme in Matrix Form}

The FD scheme can be expressed in matrix form for efficient computation. But, first, we introduce the column vectors:

\[
u^{m} = \begin{pmatrix}
u^{m}_{1} \\
\vdots \\
u^{m}_{N}
\end{pmatrix}, \quad
\mathcal{E}^{m} = \begin{pmatrix}
\mathcal{E}^{m}_{1} \\
\vdots \\
\mathcal{E}^{m}_{N}
\end{pmatrix}, \quad
\underline{f}^{m} = \begin{pmatrix}
f^{m}_{1} \\
\vdots \\
f^{m}_{N}
\end{pmatrix},
\]

and the tridiagonal $N \times N$ matrix:

\[
\textbf{G} = h^{-2} \cdot \text{tridiag}(-1, 2, -1).
\]

Then, the FD scheme $\underline{\mathcal{E}}^{m} = \theta \underline{f}^{m+1} + (1 - \theta) \underline{f}^{m}$ becomes, in matrix form. Given $\underline{u}^{0} = (u_{0}(x_{1}) \hdots u_{0}(x_{N}))^T  \in \mathbb{R}^{N}$, for $m = 0, \ldots, M-1$, find $\underline{u}^{m+1} \in \mathbb{R}^{N}$ such that:

\[
(\textbf{I} + \theta k \textbf{G}) \underline{u}^{m+1} + (-\textbf{I} + (1 - \theta) k \textbf{G}) \underline{u}^{m} = k [\theta \underline{f}^{m+1} + (1 - \theta) \underline{f}^{m}] =: \underline{F}^{m},
\]

Or equivalently:

\[
B \underline{u}^{m+1} = C \underline{u}^{m} + k \underline{F}^{m}, \quad m = 0, \ldots, M-1,
\]

where

\[
B = I + \theta kG, \quad C = -I + (1 - \theta) kG.
\]

This matrix equation allows for the efficient computation of the solution using matrix operations.

\subsubsubsection{Matlab Implementation (FDM)}

Here is a Matlab implementation of the Finite Difference Method (FDM) for solving the heat equation:

\lstinputlisting[
    language=Octave,
    caption={Code for deflection at the point of intersection},
    label={lst:heateq_fdm.m},
    frame=single,
    breaklines=true,
    numbers=left,
    stepnumber=1,
]{code/heateq_fdm.m}