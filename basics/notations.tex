Before diving into the details, let's establish the notations used throughout this tutorial. The clarity in notation is crucial for better understanding. Here are the key notations:

\begin{itemize}
    \item $X_t$: The asset price at time $t$.
    \item $T$: The maturity or exercise time of the option.
    \item $g(x)$: The option's pay-off function is a function of the underlying asset's value $x$.
    \item $\mu(t, X_t)$: The drift term in the asset price SDE.
    \item $\sigma(t, X_t)$: The volatility term in the asset price SDE.
    \item $dW_t$: The Wiener process or Brownian motion.
    \item $u(t, x)$: The option price at time $t$, given the current value of the underlying asset $X_t = x$.
    \item For $\alpha = (\alpha_1, \ldots, \alpha_d) \in \mathbb{N}_0^d$ multiindex, and $G \subset \mathbb{R}^d$ (open):
    \[|\alpha| = \sum_{i=1}^d \alpha_i.\]
    \item For smooth $u : G \rightarrow \mathbb{R}$ and $x = (x_1, \ldots, x_d) \in G$, define:
    \[D^\alpha u(x) := \frac{{\partial^{|\alpha|} u(x)}}{{\partial x_1^{\alpha_1} \cdots \partial x_d^{\alpha_d}}} = {\partial^{\alpha_1}_{x_1}} \cdots \partial^{\alpha_d}_{x_d} u(x).\]
    \item If $k = 1$, arrange the elements of $D^1 u(x) =: Du(x)$ in a vector:
    \[Du = (\nabla u)^T = \left(\frac{{\partial u}}{{\partial x_1}}, \ldots, \frac{{\partial u}}{{\partial x_d}}\right).\]
    \item If $k = 2$, arrange the elements of $D^2 u(x)$ in a matrix:
    \[D^2 u = \begin{bmatrix}
    \frac{{\partial^2 u}}{{\partial x_1^2}} & \frac{{\partial^2 u}}{{\partial x_1 \partial x_2}} & \ldots & \frac{{\partial^2 u}}{{\partial x_1 \partial x_d}} \\
    \frac{{\partial^2 u}}{{\partial x_2 \partial x_1}} & \frac{{\partial^2 u}}{{\partial x_2^2}} & \ldots & \vdots \\
    \vdots & \vdots & \ddots & \vdots \\
    \frac{{\partial^2 u}}{{\partial x_d \partial x_1}} & \ldots & \ldots & \frac{{\partial^2 u}}{{\partial x_d^2}}
    \end{bmatrix}.\]
    Please note that for brevity, I have only shown the four corner elements of the matrix, and the rest can be represented using ellipsis ($\ldots$).
    \item Write $\frac{{\partial^2 u}}{{\partial x_i \partial x_j}}$ as ${\partial_{x_i x_j} u}$. Hence, the Laplacian $\Delta u$ of $u$ can be written as:
    \[\Delta u := \sum_{i=1}^{d} \partial_{x_i x_i} u = \text{tr}(D^2 u).\]
    \item A partial differential equation (PDE) involves an unknown function of two or more variables and certain of its derivatives.
\end{itemize}

These notations will be used consistently throughout the tutorial to ensure clarity and consistency in the explanations and examples.

Now that we have established the notations let's delve deeper into the specific numerical methods for solving PDEs in the context of option pricing.
