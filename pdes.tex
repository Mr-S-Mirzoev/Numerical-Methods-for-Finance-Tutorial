\section{Partial Differential Equations (PDEs)}

In this section, we will explore the concept of Partial Differential Equations (PDEs) and their relevance in the context of option pricing. Let's start with the definition of a PDE:

\textbf{Definition (Partial Differential Equation):}
Let $k \in \mathbb{N}$. A $k$-th order PDE is an expression of the form
\[F(D^k u(x), D^{k-1} u(x), \ldots, Du(x), u(x), x) = 0, \quad x \in G,\]
where $F : \mathbb{R}^{d^k} \times \mathbb{R}^{d^{k-1}} \times \ldots \times \mathbb{R}^d \times \mathbb{R} \times G \rightarrow \mathbb{R}$ is a given function, $u : G \rightarrow \mathbb{R}$ is the unknown function, and $G$ represents the domain of the PDE.

In other words, a PDE is an equation involving an unknown function $u$ of two or more variables and certain of its derivatives up to the $k$-th order.

PDEs play a fundamental role in various fields, including physics, engineering, and finance. In the context of option pricing, PDEs are used to model the dynamics of financial instruments and derive the corresponding option pricing equations.

In the upcoming sections, we will delve deeper into specific numerical methods for solving PDEs and their application to option pricing problems.
