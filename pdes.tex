\section{Partial Differential Equations (PDEs)}

This section will explore the concept of Partial Differential Equations (PDEs) and their relevance in option pricing. Let's start with the definition of a PDE:

\textbf{Definition (Partial Differential Equation):}
Let $k \in \mathbb{N}$. A $k$-th order PDE is an expression of the form
\[F(D^k u(x), D^{k-1} u(x), \ldots, Du(x), u(x), x) = 0, \quad x \in G,\]
where $F : \mathbb{R}^{d^k} \times \mathbb{R}^{d^{k-1}} \times \ldots \times \mathbb{R}^d \times \mathbb{R} \textcolor{red}{\times G} \rightarrow \mathbb{R}$ is a given function, $u : G \rightarrow \mathbb{R}$ is the unknown function, and $G$ represents the domain of the PDE.

In other words, a PDE is an equation involving an unknown function $u$ of two or more variables and certain of its derivatives up to the $k$-th order.

PDEs are fundamental in various fields, including physics, engineering, and finance. For example, in the context of option pricing, PDEs are used to model the dynamics of financial instruments and derive the corresponding option pricing equations.

\subsection{Important Examples of PDEs}

Here are some crucial examples of PDEs:

\textbf{Poisson Equation:}
Given a function $f : G \rightarrow \mathbb{R}$, find $u : G \rightarrow \mathbb{R}$ such that
\[-\Delta u(x) = -\partial_{x_1 x_1} u(x) - \partial_{x_2 x_2} u(x) - \ldots - \partial_{x_d x_d} u(x) = f(x)\]
The associated function $F(y^2, y^1, y^0, x) := -\sum_{j=1}^d y^{2}_{j j} - f(x)$.

$f$ is "smooth” enough thus, $u$ exists and is unique, \textcolor{red}{provided we specify
additional conditions on the boundary}.

\textbf{Heat Equation:}
Given $f = f(t, x)$, seek $u = u(t, x)$ such that
\[\partial_t u - \Delta u(x) = f(x, t)\]

The associated function $F(y^2, y^1, y^0, (t, x)) := y^1_0 - \sum_{j=1}^d y^{2}_{j j} - f(x, t)$.

Since the PDE is first order in t, we need an \textcolor{red}{initial condition for t = 0}.

These are just a few examples of PDEs commonly encountered in various mathematical and scientific disciplines. For example, in option pricing, PDEs play a crucial role in formulating and solving pricing equations for derivative instruments.

In the upcoming sections, we will delve deeper into specific numerical methods for solving PDEs and their application to option pricing problems.

\subsection{Types of PDEs}

Let's consider a linear 2nd-order PDE in $d + 1$ variables. The function $F$ has the form
\[F(D^2u, Du, u, x) = -\sum_{i,j=0}^d a_{ij}(x) \partial_{x_i x_j} u + \sum_{i=0}^d b_i(x) \partial_{x_i} u + c(x) u - f(x).\]
Here, $a_{ij}$, $b_i$, $c$, and $f$ are given real-valued functions, and $A(x) = (a_{ij}(x))_{i,j=0}^d$ is a symmetric matrix with real eigenvalues $\lambda_0(x) \leq \lambda_1(x) \leq \ldots \leq \lambda_d(x)$.

We can classify the PDE based on its properties:

\textbf{Elliptic PDE:}
The PDE is called elliptic if $\lambda_i(x) \neq 0$ for all $i$ and $\mathrm{sign}(\lambda_0(x)) = \ldots = \mathrm{sign}(\lambda_d(x))$.

\textbf{Parabolic PDE:}
The PDE is called parabolic if there exists a unique $j \in \{0, \ldots, d\}$ such that $\lambda_j(x) = 0$ and $\mathrm{rank}(A(x), b(x)) = d + 1$.

\textbf{Hyperbolic PDE:}
The PDE is called hyperbolic if $\lambda_i(x) \neq 0$ for all $i$ and there exists a unique $j \in \{0, \ldots, d\}$ such that $\mathrm{sign}(\lambda_j(x)) \neq \mathrm{sign}(\lambda_k(x))$ for $k \in \{0, \ldots, d\} \setminus \{j\}$.

The PDE is classified as elliptic, parabolic, or hyperbolic on $G$ if it exhibits the respective properties for all $x \in G$.

Understanding the type of PDE is essential for selecting appropriate numerical methods for solving them.

In the upcoming sections, we will delve deeper into specific numerical methods for solving PDEs and their application to option pricing problems.